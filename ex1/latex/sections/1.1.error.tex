\documentclass[preview,10pt,border=8pt]{standalone}
%Citation
	\usepackage[sort&compress,numbers]{natbib}
	\newcommand{\supercite}[1]{\textsuperscript{\,%
		[\citenum{#1}]}}
	\let\fancycite\cite
	\renewcommand{\cite}[1]{\textup{\fancycite{#1}}}
	\renewcommand\refname{参考文献}
%Chinese
	\usepackage[UTF8,heading=false,fontset=fandol]{ctex}
	\xeCJKsetup{underdot = {
		boxdepth=0pt, format=\huge, depth=.4em
	}}
	\newcommand{\cjkdot}[1]{\CJKunderdot{#1}}
%ParagraphSetting
	\setlength{\parskip}{.3\baselineskip}
	\usepackage[defaultlines=2,all]{nowidow}
	\postdisplaypenalty=300
%Presettings
	\usepackage[table]{xcolor}
	\usepackage{graphicx}
	\usepackage[font={sf}]{caption}
	\usepackage[above]{placeins}
%MathSetting
	\let\latexointop\ointop
	\usepackage{amsmath,amssymb,nicefrac,extarrows,eqnarray}%amsthm
	\usepackage{bm,mathrsfs,mathtools,physics}
	\usepackage{wasysym,stmaryrd,esint,siunitx}
	\usepackage{enumitem,stackengine,titling,varwidth}
	\usepackage{tikz,tikz-cd,tkz-euclide}
	\usepackage{resizegather,empheq}
	\usetagform{default}
	\usepackage{calligra,romannum,dsfont,fourier-orns}
	\usepackage{ccicons}
	\let\ointop\undefined
	\let\ointop\latexointop%oint unchanged by esint
	\DeclareMathAlphabet{\mathcalligra}{T1}{calligra}{m}{n}
	\DeclareFontShape{T1}{calligra}{m}{n}{<->s*[2.2]callig15}{}
	\DeclarePairedDelimiter\ave{\langle}{\rangle}
	\DeclarePairedDelimiterX\inprod[2]{\langle}{\rangle}{#1,#2}
	\newcommand{\scriptr}{\mathcalligra{r}\,}
	\newcommand{\rvector}{\pmb{\mathcalligra{r}}\,}
	\newcommand\inlineeqno{\stepcounter{equation}\ (\theequation)}
	\newcommand{\mbb}[1]{\mathbb{#1}}
	\newcommand{\mrm}[1]{\mathrm{#1}}
	\newcommand{\mcal}[1]{\mathcal{#1}}
	\newcommand{\tup}[1]{\textup{#1}}
	\newcommand*\bbox[1]{\fbox{\hspace{1em}\addstackgap[5pt]{#1}\hspace{1em}}}
	\newcommand\scalemath[2]{\scalebox{#1}{\mbox{\ensuremath{\displaystyle #2}}}}
	\newcommand\raisemath[2]{\raisebox{#1\depth}{${#2}$}}
	\empheqset{box=\bbox}
	\numberwithin{equation}{section}
%PageSetting
	\usepackage{array,booktabs,tabularx}
	\usepackage[colorlinks=true,linkcolor=blue]{hyperref}
	\usepackage[vmargin=4cm,hmargin=3cm,%
		footnotesep=\baselineskip]{geometry}
	\usepackage[bottom,splitrule]{footmisc}
	\newcolumntype{C}[1]{>{\hsize=#1\hsize%
		\centering\arraybackslash}X}
	\setlist{itemsep=0pt,topsep=0pt,labelindent=\parindent,leftmargin=0pt,itemindent=*}
	\setlength{\footnotesep}{2.5\parskip}
	\newfontfamily\signature{Vladimir Script}
	\newcommand{\newparagraph}{\pagebreak[3]\noindent%
	%\hrulefill%
	\hfil
	~\raisebox{-4pt}[10pt][10pt]{\leafright~\signature \qquad~\leafleft}~ %
	%\hrulefill%
	\par
	}
%Specials
	\newcommand{\hodgedual}{\operatorname{\star}}
	\newcommand{\dual}{\ \xlongleftrightarrow{\ \textrm{dual}\ }\ }
	\newcommand{\idty}{\mathds{1}}

% Keep parindent
	\usepackage{etoolbox}
	\edef\parindentkept{\the\parindent}
	\edef\parskipkept{\the\parskip}
	\patchcmd{\preview}
	  {\ignorespaces} %%% \preview ends with \ignorespaces
	  {\parindent\parindentkept%
	  \parskip\parskipkept%
	  \ignorespaces}
	  {}{}
% Standalone extras
	\newcommand{\morewhite}{\vspace{16\baselineskip}
		\newparagraph}
	\newcommand{\onelevelup}{../}

\addtocounter{section}{1-1}
\begin{document}
\section{数值误差的避免}
\subsection{求平均的误差}
	$N$数平均的误差来源于求和、除以$N$两个过程;在$N$较大时,除以大数所引入的误差相对较小,此时求和的误差占主要成分。
	
	两数相加时,引入的\textit{相对误差}为机器精度$\frac{\epsilon_M}{2}$; 记$x_0 = \max\abs{x_i}$, 考虑最坏的情况,即可能的误差最大值,这一情形在每个$x_i \to x_0$时取到。不妨设$x_i$均为正数,此时求和的上限为:
	\begin{equation}
		f\circ f\circ\dots\circ f(x_0)
		\equiv f^{N-1}\circ(x_0),\quad
		f(x) = (x + x_0)\,
			\pqty{1 + \frac{\epsilon_M}{2}}
	\end{equation}
	这里$f$是每次数值求和操作的函数表示。
	
	\begin{table}[!h]
	\centering
	
	\begin{tabularx}{.6\linewidth}{C{.8}|C{1}C{1}%
		@{\hspace{-.8em}}}
	\toprule
		\textit{作用于$x_0$} &
		\textit{$\order{1}$项} &
		\textit{$\order{\frac{\epsilon_M}{2}}$系数} \\
	\midrule
		$f^0 = \idty$ & $x_0$        & 0        \\ 
		$f^1$         & $2x_0$       & $2x_0$   \\
		$f^2$         & $3x_0$       & $5x_0$   \\
		$\vdots$      & $\vdots$     & $\vdots$ \\
		$f^k$         & $x_0 + kx_0$ & $c_k$    \\
	\bottomrule
	\end{tabularx}
	\caption*{$f$的作用规律:先加$x_0$, 再乘以$\pqty{1 + \frac{\epsilon_M}{2}}$}
	\end{table}
	
	考察$\frac{\epsilon_M}{2}$的系数,设$f^k$作用后的$\frac{\epsilon_M}{2}$系数为$c_k$, 则不难发现:
	\begin{equation}
		c_k = c_{k-1} + x_0 + kx_0
	\end{equation}
	其中$kx_0$源于前一步$\order{1}$项的系数。已知$c_0 = 0$, 展开此递推关系,即得:
	\begin{equation}
	\begin{gathered}
		c_{N-1} = \frac{(N+2)(N-1)}{2}\,x_0,\\[1.5ex]
		\textit{均值的误差限:}\ 
		\frac{1}{N}\cdot c_{N-1} \frac{\epsilon_M}{2}
		= \frac{(N+2)(N-1)}{2N} \frac{\epsilon_M}{2}
			\max\abs{x_i}
		\sim \frac{N}{2} \frac{\epsilon_M}{2}
			\max\abs{x_i}
	\end{gathered}
	\end{equation}
\subsection{方差计算的稳定性}
	两种方差计算公式如下:
	\begin{subequations}
	\begin{align}
		S^2 &= \frac{1}{N - 1}
			\Bqty\bigg{\sum_i x_i^2 - N\bar{x}^2}%
		\label{eq:sum&substract} \\
		&= \frac{1}{N - 1}
			\sum_i (x_i - \bar{x})^2
		\label{eq:shiftedSum}
	\end{align}
	\end{subequations}
	沿用前文给出的估计办法,可以给出两式的误差限;有:
	\begin{equation}
	\begin{aligned}
		e_{(a)} &\sim S^2\,\frac{\epsilon_M}{2}
			+ \Bqty{\frac{N}{2} \frac{\epsilon_M}{2}
					\max\abs{x_i}^2
				+ \frac{N}{2} \frac{\epsilon_M}{2}
					\max\abs{x_i}\times 2} \\
		&= \pqty{S^2
			+ \frac{N}{2} \max\abs{x_i}^2
			+ N \max\abs{x_i}}\,
		\frac{\epsilon_M}{2}, \\[1.2ex]
		e_{(b)}
		&\sim \frac{N}{2} \frac{\epsilon_M}{2}
			\max\abs{x_i - \bar{x}}^2
	\end{aligned}
	\end{equation}
	
	可见,多数情况下第一式 \eqref{eq:sum&substract} 将带来较大误差;特别是在$x_i$很大但方差却很小的情况下,此时将产生\textit{大数相消},从而大量损失有效数字。相比之下,第二式 \eqref{eq:shiftedSum} 较为稳定和准确。
\subsection{递归计算的稳定性}
	考察
		$I_n = \displaystyle\int_0^1
			\dd{x} \frac{x^n}{x+5}$, 
	首先有$I_0 = \ln\,(x+5) \big|_0^1 = \ln\frac{6}{5}$, 而:
	\begin{equation}
		I_k + 5I_{k-1}
		= \int_0^1 \dd{x}\,
			\frac{x^k + 5x^{k-1}}{x+5}
		= \int_0^1 \dd{x} x^{k-1}
		= \frac{x^k}{k} \bigg|_0^1
		= \frac{1}{k},\quad
		k = 1,2,\dots
	\end{equation}
	从而可以递归地给出$I_k$的值。
	
	关注这一过程的误差传递,设计算值
		$\hat{I}_{k-1} = I_{k-1} + \epsilon_{k-1}$,
	则相应地:
	\begin{equation}
	\begin{aligned}
		\hat{I}_k &\sim \pqty{\frac{1}{k}
			- 5\hat{I}_{k-1}}\,
		\pqty{1 + \frac{\epsilon_M}{2}} \\[.5ex]
		&\sim I_k - 5\epsilon_{k-1}
			+ I_k \frac{\epsilon_M}{2} \\[.8ex]
		\textit{即有:}\,
		\epsilon_{k}
		&\sim - \pqty{\,\text{\Large
				$\nicefrac{5}{I_k}$}}\,
			\epsilon_{k-1} + \frac{\epsilon_M}{2}
	\end{aligned}
	\end{equation}
	系数$\kappa = \abs{{5}/{I_k}}$是关键;若$\kappa < 1$, 则误差将得到控制,不会进一步放大。
	
	然而,不幸的是,本问题中的
		$I_n < I_0 < 1$, 即始终有$\kappa > 1$, 
	初始误差$\epsilon$将随递归过程不断(指数)放大,可见这一算法是不稳定的。
%\morewhite
\end{document}
