\documentclass[preview,10pt,border=8pt]{standalone}
%Citation
	\usepackage[sort&compress,numbers]{natbib}
	\newcommand{\supercite}[1]{\textsuperscript{\,%
		[\citenum{#1}]}}
	\let\fancycite\cite
	\renewcommand{\cite}[1]{\textup{\fancycite{#1}}}
	\renewcommand\refname{参考文献}
%Chinese
	\usepackage[UTF8,heading=false,fontset=fandol]{ctex}
	\xeCJKsetup{underdot = {
		boxdepth=0pt, format=\huge, depth=.4em
	}}
	\newcommand{\cjkdot}[1]{\CJKunderdot{#1}}
%ParagraphSetting
	\setlength{\parskip}{.3\baselineskip}
	\usepackage[defaultlines=2,all]{nowidow}
	\postdisplaypenalty=300
%Presettings
	\usepackage[table]{xcolor}
	\usepackage{graphicx}
	\usepackage[font={sf}]{caption}
	\usepackage[above]{placeins}
%MathSetting
	\let\latexointop\ointop
	\usepackage{amsmath,amssymb,nicefrac,extarrows,eqnarray}%amsthm
	\usepackage{bm,mathrsfs,mathtools,physics}
	\usepackage{wasysym,stmaryrd,esint,siunitx}
	\usepackage{enumitem,stackengine,titling,varwidth}
	\usepackage{tikz,tikz-cd,tkz-euclide}
	\usepackage{resizegather,empheq}
	\usetagform{default}
	\usepackage{calligra,romannum,dsfont,fourier-orns}
	\usepackage{ccicons}
	\let\ointop\undefined
	\let\ointop\latexointop%oint unchanged by esint
	\DeclareMathAlphabet{\mathcalligra}{T1}{calligra}{m}{n}
	\DeclareFontShape{T1}{calligra}{m}{n}{<->s*[2.2]callig15}{}
	\DeclarePairedDelimiter\ave{\langle}{\rangle}
	\DeclarePairedDelimiterX\inprod[2]{\langle}{\rangle}{#1,#2}
	\newcommand{\scriptr}{\mathcalligra{r}\,}
	\newcommand{\rvector}{\pmb{\mathcalligra{r}}\,}
	\newcommand\inlineeqno{\stepcounter{equation}\ (\theequation)}
	\newcommand{\mbb}[1]{\mathbb{#1}}
	\newcommand{\mrm}[1]{\mathrm{#1}}
	\newcommand{\mcal}[1]{\mathcal{#1}}
	\newcommand{\tup}[1]{\textup{#1}}
	\newcommand*\bbox[1]{\fbox{\hspace{1em}\addstackgap[5pt]{#1}\hspace{1em}}}
	\newcommand\scalemath[2]{\scalebox{#1}{\mbox{\ensuremath{\displaystyle #2}}}}
	\newcommand\raisemath[2]{\raisebox{#1\depth}{${#2}$}}
	\empheqset{box=\bbox}
	\numberwithin{equation}{section}
%PageSetting
	\usepackage{array,booktabs,tabularx}
	\usepackage[colorlinks=true,linkcolor=blue]{hyperref}
	\usepackage[vmargin=4cm,hmargin=3cm,%
		footnotesep=\baselineskip]{geometry}
	\usepackage[bottom,splitrule]{footmisc}
	\newcolumntype{C}[1]{>{\hsize=#1\hsize%
		\centering\arraybackslash}X}
	\setlist{itemsep=0pt,topsep=0pt,labelindent=\parindent,leftmargin=0pt,itemindent=*}
	\setlength{\footnotesep}{2.5\parskip}
	\newfontfamily\signature{Vladimir Script}
	\newcommand{\newparagraph}{\pagebreak[3]\noindent%
	%\hrulefill%
	\hfil
	~\raisebox{-4pt}[10pt][10pt]{\leafright~\signature \qquad~\leafleft}~ %
	%\hrulefill%
	\par
	}
%Specials
	\newcommand{\hodgedual}{\operatorname{\star}}
	\newcommand{\dual}{\ \xlongleftrightarrow{\ \textrm{dual}\ }\ }
	\newcommand{\idty}{\mathds{1}}

% Keep parindent
	\usepackage{etoolbox}
	\edef\parindentkept{\the\parindent}
	\edef\parskipkept{\the\parskip}
	\patchcmd{\preview}
	  {\ignorespaces} %%% \preview ends with \ignorespaces
	  {\parindent\parindentkept%
	  \parskip\parskipkept%
	  \ignorespaces}
	  {}{}
% Show footnote
	\let\purefootnote\footnote
	\newcommand{\showfootnote}{}
	\renewcommand{\footnote}[1]{%
		\purefootnote{#1}%
		\renewcommand{\showfootnote}{{%
			\noindent\footnotesize%
			\lefthand\ 
			[\number\value{footnote}]\ 
			\textit{#1}}%
	}}
% Standalone extras
	\newcommand{\morewhite}{\vspace{16\baselineskip}
		\newparagraph}
	\newcommand{\onelevelup}{../}

\addtocounter{section}{3-1}
\begin{document}
\section{Hilbert矩阵}
\subsection{$H_n$的具体形式}
	多项式近似的残差
		$D = \displaystyle\int_0^1 \dd{x}
			\Bqty{
				\sum_{i=1}^n c_i x^{i-1}
				- f(x)}^{\!2}$,
	为使其尽可能小,对参量$c_j$微分,即有:
	\begin{equation}
	\begin{aligned}
		0 = \pdv{D}{c_j}
		&= \displaystyle\int_0^1 \dd{x}\,
			\pdv{c_j}\Bqty{
				\sum_{i=1}^n c_i x^{i-1}
					- f(x)}^{\!2} \\[.5ex]
		&= \displaystyle\int_0^1 \dd{x}\cdot
			2\,\Bqty{
				\sum_{i=1}^n c_i x^{i-1}
					- f(x)}\,
				\sum_{i=1}^n \delta^j_i x^{i-1} \\[.5ex]
		&= \displaystyle\int_0^1 \dd{x}\cdot
			2x^{j-1}\Bqty{
				\sum_{i=1}^n c_i x^{i-1}
					- f(x)}
	\end{aligned}
	\end{equation}
	积分,得
		$0 = \displaystyle\sum_{i=1}^n c_i\,
				\frac{1}{i+j-1}
			- \int_0^1 \dd{x}
				x^{j-1} f(x)$,
	即有:
	\begin{equation}
	\begin{gathered}
		H_n\cdot c = b,\\
		(H_n)_{ij} = \frac{1}{i+j-1},\quad
		b_j = \displaystyle\int_0^1 \dd{x}
			x^{j-1} f(x)
	\end{gathered}
	\end{equation}
\subsection{$H_n$的特征}
	由$(H_n)_{ij} = \frac{1}{i+j-1} = (H_n)_{ji}$, 可见$H_n$为对称矩阵;此外,参见上文,有:
	\begin{equation}
		c^T H_n c = \sum_{i,j=1}^n c_i c_j
			\int_0^1 x^{i+j-2} \dd{x}
		= \int_0^1 \dd{x}
			\Bqty{\sum_{i=1}^n
				c_i x^{i-1}}^{\!2} \ge 0
	\end{equation}
	当且仅当
		$\displaystyle\sum_{i=1}^n
			c_i x^{i-1} \equiv 0$
	即$c = 0$时取到等号。可见Hilbert矩阵是对称正定矩阵。
	
	此外,由对称正定性还可知Hilbert矩阵必定是非奇异的。事实上,对任一对称正定矩阵$A$而言,若它同时是奇异矩阵,则$\det A = 0$, 故存在$c \ne 0$使得$Ac = 0$, 进而导致$c^T A c = 0$, 这与对称正定性矛盾。因此对称正定矩阵均非奇异。
\subsection{$\det H_n$的行为}
	已知:
	\begin{equation}
		\det H_n = \frac{c_n^4}{c_{2n}},\quad
			c_n = 1!\cdot 2! \dots (n-1)!
	\end{equation}
	为估计$\det H_n$的大小,取对数,注意到
		$\ln c_n
			= \displaystyle\sum_{m=1}^{n-1}
				\ln m!$\,, 可得:
	\begin{equation}
		\ln\det H_n
		= 4\ln c_n - \ln c_{2n}
		= \Bqty\bigg{4\sum_{m=1}^{n-1}
			- \sum_{m=1}^{2n-1}\,}\,\ln m!
%		= 3\ln c_n - \ln \frac{c_{2n}}{c_n}
%		= \Bqty\bigg{3\sum_{m=1}^{n-1}
%			- \sum_{m=n}^{2n-1}\,}\,\ln m!
	\end{equation}
	
	下面尝试用Stirling近似给出上式子的一个近似。已知:
	\begin{equation}
		m! \sim \sqrt{2\pi m}\,
			\pqty{\frac{m}{e}}^{\!m}
	\end{equation}
	带入上式,初步化简后得到:
	\begin{equation}
	\begin{aligned}
		\ln\det H_n
		&\sim \Bqty\bigg{4\sum_{m=1}^{n-1}
			- \sum_{m=1}^{2n-1}\,}\,m\ln m \\
		&\qquad + n + \pqty{n - \frac{3}{2}}\,
			\ln\,(2\pi)\,
			+ \frac{1}{2}\,\pqty\big{
				4\ln\,(n-1)! - \ln\,(2n-1)!\,} \\[1ex]
%		&\sim \Bqty\bigg{3\sum_{m=1}^{n-1}
%			- \sum_{m=n}^{2n-1}\,}\,m\ln m \\
%		&\qquad + n + \pqty{n - \frac{3}{2}}\,
%				\ln\,(2\pi) \\
%		&\qquad + 2\,(n-1)\,\ln\,(n-1)
%			- \frac{1}{2}\,(2n-1)\,\ln\,(2n-1) \\
%		&\qquad\qquad - \pqty{n - \frac{3}{2}}
%			- \frac{3}{4} \ln\,(2\pi)
%			+ \ln\,(n-1)
%				- \frac{1}{4}\ln\,(2n-1) \\[1ex]
%		&= \Bqty\bigg{3\sum_{m=1}^{n-1}
%			- \sum_{m=n}^{2n-1}\,}\,m\ln m \\
%		&\qquad + \frac{3}{2} + \pqty{n - \frac{9}{4}}\,
%				\ln\,(2\pi) \\
%		&\qquad + 2\,(n-1)\,\ln\,(n-1)
%			- \frac{1}{2}\,(2n-1)\,\ln\,(2n-1) \\
%		&\qquad\qquad
%			+ \ln\,(n-1)
%				- \frac{1}{4}\ln\,(2n-1) \\[1ex]
		&\sim \Bqty\bigg{4\sum_{m=1}^{n-1}
			- \sum_{m=1}^{2n-1}\,}\,m\ln m \\
		&\qquad + (2n-1)\,\ln\,(n-1)
			- \pqty{n-\frac{1}{4}}\,\ln\,(2n-1) \\
		&\qquad\qquad
			+ \pqty{n - \frac{9}{4}}\,
				\ln\,(2\pi) + \frac{3}{2}
	\end{aligned}
	\end{equation}%
\pagebreak[3]
	
	进一步,考虑积分的几何意义,对$\sum m\ln m$可采用如下近似:
	\begin{equation}
		\sum_{m=1}^k m\ln m
		\sim \int_{\frac{3}{2}}^{n + \frac{1}{2}}
			\dd{x} x\ln x
	\end{equation}
	即可得到$\ln\det H_n$的完整近似形式。截取最高阶项,我们得到:
	\begin{empheq}{equation}
		\ln\det H_n \sim - 2n^2 \ln 2,\quad
		\det H_n \sim 4^{-n^2},\quad
		n\to\infty
	\end{empheq}
	
	由此可见,$\det H_n$随$n$增大而指数地减小,即$H_n$迅速地接近于一个奇异矩阵。注意,$4^{-n^2}$给出的粗略近似(后面标记为\textsf{Rough})适用于大宗量$n$的情形;对于较小的$n$值而言,由于略去了过多的低阶项,结果可能不甚理想。
%\morewhite
\end{document}
