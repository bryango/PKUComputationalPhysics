\documentclass[preview,10pt,border=8pt]{standalone}
%Citation
	\usepackage[sort&compress,numbers]{natbib}
	\newcommand{\supercite}[1]{\textsuperscript{\,%
		[\citenum{#1}]}}
	\let\fancycite\cite
	\renewcommand{\cite}[1]{\textup{\fancycite{#1}}}
	\renewcommand\refname{参考文献}
%Chinese
	\usepackage[UTF8,heading=false,fontset=fandol]{ctex}
	\xeCJKsetup{underdot = {
		boxdepth=0pt, format=\huge, depth=.4em
	}}
	\newcommand{\cjkdot}[1]{\CJKunderdot{#1}}
%ParagraphSetting
	\setlength{\parskip}{.3\baselineskip}
	\usepackage[defaultlines=2,all]{nowidow}
	\postdisplaypenalty=300
%Presettings
	\usepackage[table]{xcolor}
	\usepackage{graphicx}
	\usepackage[font={sf}]{caption}
	\usepackage[above]{placeins}
%MathSetting
	\let\latexointop\ointop
	\usepackage{amsmath,amssymb,nicefrac,extarrows,eqnarray}%amsthm
	\usepackage{bm,mathrsfs,mathtools,physics}
	\usepackage{wasysym,stmaryrd,esint,siunitx}
	\usepackage{enumitem,stackengine,titling,varwidth}
	\usepackage{tikz,tikz-cd,tkz-euclide}
	\usepackage{resizegather,empheq}
	\usetagform{default}
	\usepackage{calligra,romannum,dsfont,fourier-orns}
	\usepackage{ccicons}
	\let\ointop\undefined
	\let\ointop\latexointop%oint unchanged by esint
	\DeclareMathAlphabet{\mathcalligra}{T1}{calligra}{m}{n}
	\DeclareFontShape{T1}{calligra}{m}{n}{<->s*[2.2]callig15}{}
	\DeclarePairedDelimiter\ave{\langle}{\rangle}
	\DeclarePairedDelimiterX\inprod[2]{\langle}{\rangle}{#1,#2}
	\newcommand{\scriptr}{\mathcalligra{r}\,}
	\newcommand{\rvector}{\pmb{\mathcalligra{r}}\,}
	\newcommand\inlineeqno{\stepcounter{equation}\ (\theequation)}
	\newcommand{\mbb}[1]{\mathbb{#1}}
	\newcommand{\mrm}[1]{\mathrm{#1}}
	\newcommand{\mcal}[1]{\mathcal{#1}}
	\newcommand{\tup}[1]{\textup{#1}}
	\newcommand*\bbox[1]{\fbox{\hspace{1em}\addstackgap[5pt]{#1}\hspace{1em}}}
	\newcommand\scalemath[2]{\scalebox{#1}{\mbox{\ensuremath{\displaystyle #2}}}}
	\newcommand\raisemath[2]{\raisebox{#1\depth}{${#2}$}}
	\empheqset{box=\bbox}
	\numberwithin{equation}{section}
%PageSetting
	\usepackage{array,booktabs,tabularx}
	\usepackage[colorlinks=true,linkcolor=blue]{hyperref}
	\usepackage[vmargin=4cm,hmargin=3cm,%
		footnotesep=\baselineskip]{geometry}
	\usepackage[bottom,splitrule]{footmisc}
	\newcolumntype{C}[1]{>{\hsize=#1\hsize%
		\centering\arraybackslash}X}
	\setlist{itemsep=0pt,topsep=0pt,labelindent=\parindent,leftmargin=0pt,itemindent=*}
	\setlength{\footnotesep}{2.5\parskip}
	\newfontfamily\signature{Vladimir Script}
	\newcommand{\newparagraph}{\pagebreak[3]\noindent%
	%\hrulefill%
	\hfil
	~\raisebox{-4pt}[10pt][10pt]{\leafright~\signature \qquad~\leafleft}~ %
	%\hrulefill%
	\par
	}
%Specials
	\newcommand{\hodgedual}{\operatorname{\star}}
	\newcommand{\dual}{\ \xlongleftrightarrow{\ \textrm{dual}\ }\ }
	\newcommand{\idty}{\mathds{1}}

% Keep parindent
	\usepackage{etoolbox}
	\edef\parindentkept{\the\parindent}
	\edef\parskipkept{\the\parskip}
	\patchcmd{\preview}
	  {\ignorespaces} %%% \preview ends with \ignorespaces
	  {\parindent\parindentkept%
	  \parskip\parskipkept%
	  \ignorespaces}
	  {}{}
% Standalone extras
	\newcommand{\morewhite}{\vspace{16\baselineskip}
		\newparagraph}
	\newcommand{\onelevelup}{../}

\addtocounter{section}{2-1}
\begin{document}
\section{矩阵的模与条件数}
\subsection{矩阵$A$的基本性质}
	考虑矩阵$A$, 有:
	\begin{equation}
		\pqty{A - \idty}_{ij} =
		\bigg\lbrace
		\begin{aligned}\,
			& {-1}, \,
			&&\text{for}\ \ i < j, \\[-.5ex]
			& 0, 
			&&\text{for}\ \ i \ge j,
		\end{aligned}
	\end{equation}
	计算$n$阶$A_n$的行列式,注意有递归关系
		$ A_{n+1} = \pqty{
		 	\begin{smallmatrix}
				1 & [-1] \times n\ \\[.5ex]
				\qquad & A_{n}
		 	\end{smallmatrix}
		}$\,
	——\ 这里借用了python的记号:$[-1] \times n$表示长为$n$的常数列表。按第一行展开(Laplace expansion),注意到$A_{1j}$元素的\textit{代数余子式}(minor)均有一列零元,故其对行列式的贡献为零,从而:
	\begin{equation}
		\det A_n = 1\times\det A_{n-1}
		= \dots = \det A_1 = 1
	\end{equation}
	
	事实上,上述过程可推广到任何三角矩阵,由此得到三角矩阵的本征值:
	\begin{equation}
		\det A_n = a_{11} \det A_{n-1}
		= \dots = \prod_i a_{ii}
	\end{equation}
	即其对角元素的乘积。
\subsection{$A^{-1}$的形式}
	承接上文,记$A_n$的某代数余子式为$\abs{\tilde{A}^{(n)}_{ij}}$, $\abs{\,\cdot\,}$为行列式的简记符号;从定义出发,有:
	\begin{equation}
		A^{-1} = \frac{\operatorname{adj} A}{\det A}
	\end{equation}
	其中$\operatorname{adj} A$的元素为
		$(-1)^{i+j}\,
			\abs*{\tilde{A}_{ji}} / \abs{A}$, 
	注意指标有交换,即应当取一个额外的转置。
	
	类似前面的讨论,对一般的上三角矩阵,均有:
	\begin{equation}
		\abs{\tilde{A}^{(n)}_{ij}} =
		\bigg\lbrace
		\begin{aligned}\,
			& 0, \,
			&&\text{for}\ \ i < j, \\[-.5ex]
			& \abs{A_n} / a_{ii}, 
			&&\text{for}\ \ i = j,
		\end{aligned}
	\end{equation}
	再复合上一个转置,可得上(下)三角矩阵的逆依然是上(下)三角矩阵。
	
	进一步,$\abs{\tilde{A}^{(n)}_{ij}},\,i>j$的情形较为复杂,这里同样采用递归的办法。不难发现,有:
	\begin{equation}
	\begin{aligned}
	\abs{\tilde{A}^{(n+1)}_{ij}}
	&= \left|\ 
	 	\begin{matrix}
			1 & [-1] \times (n-1)\ \\[.8ex]
			\qquad & \tilde{A}^{(n)}_{i-1,j-1}
	 	\end{matrix}\ \right|
	 = \abs{\tilde{A}^{(n)}_{i-1,j-1}},\quad
	 	i > j > 1,\\[1ex]
	&= \left|\ 
	 	\begin{matrix}
			\tilde{A}^{(n)}_{i,j} & \vdots \\[.8ex]
			& 1 
	 	\end{matrix}\ \right|
	 = \abs{\tilde{A}^{(n)}_{ij}},\quad
	 	i > j,\\
	\end{aligned}
	\end{equation}
	上述化简利用了$A$的上三角特性。
	
	由此可见,$A^{-1}$具有平行于主对角线的带状结构,且:
	\begin{equation}
		A^{-1}_n = \pqty\Bigg{\ 
		 	\begin{matrix}
				A^{-1}_{n-1} & \vdots \\[.8ex]
				& 1
		 	\end{matrix}\ \ }
	 	= \pqty\Bigg{\ \ 
 		 	\begin{matrix}
 				1 & \cdots \\[.8ex]
 				& A^{-1}_{n-1}
 		 	\end{matrix}\ \ }
	\end{equation}
	综上,$A^{-1}$的形态已经基本确定,唯一未定的元素只剩下最右上角的$A^{-1}_{1,n}$了。可以方便地以待定系数的方法给出$A^{-1}_{1,n}$; 利用$A^{-1}A = \idty$, 不难得到:
	\begin{equation}
		A^{-1}_{1,n}
		= \sum_{j<n} A^{-1}_{1,j}
		= \sum_{i>1} A^{-1}_{i,n}
	\end{equation}
	即它是第1行(或第$n$列)除去其自身以外其他元素的总和。如此,便可以递归地得到:
	\begin{equation}
	\begin{gathered}
		A^{-1}_1 = (\,1\,),\quad
		A^{-1}_2 = \pqty{
			\begin{array}{cc}
				1 & 1 \\
				  & 1
			\end{array}
		},\quad
		A^{-1}_3 = \pqty{
			\begin{array}{ccc}
				1 & 1 & 2 \\
				  & 1 & 1 \\
				  &   & 1
			\end{array}
		},\quad
		A^{-1}_4 = \pqty{
			\begin{array}{cccc}
				1 & 1 & 2 & 4 \\
				  & 1 & 1 & 2 \\
				  &   & 1 & 1 \\
				  &   &   & 1 \\
			\end{array}
		},\quad\cdots\\
		A^{-1}_{ij} = \left\lbrace
		\begin{aligned}\,
			& 0, \,
			&&\text{for}\ \ i > j, \\[-.5ex]
			& 1, \,
			&&\text{for}\ \ i = j, \\[-.5ex]
			& 2^{j-i-1}, 
			&&\text{for}\ \ i < j,
		\end{aligned}\right.
	\end{gathered}
	\end{equation}
\subsection{矩阵的$\infty$模}
	已知矢量$p$模:
	\begin{equation}
		\norm{x}_p
		= \Bqty\bigg{\sum_i \abs{x_i}^p}^\frac{1}{p}
		\xrightarrow{\ p\to\infty\ }
		\max\abs{x_i}\,\lim\limits_{p\to\infty}
			\Bqty{1 + \sum_{
					\abs{x_i} < \abs{x_{\max}}}\ 
				\abs{\frac{x_i}{x_{\max}}}^p\,
			}^\frac{1}{p}
		= \max\abs{x_i}
	\end{equation}
	考虑相应的矩阵模,首先有:
	\begin{equation}
		\frac{\norm{Ax}_p}{\norm{x}_p}
		\xrightarrow{\ p\to\infty\ }
		\frac{\max\abs{A_{ij} x^j}}{\max\abs{x_i}}
	\end{equation}
	
	$\forall\, x\ne 0$, 取上界,即得到$\norm{A}$. 注意数乘不改变$\frac{\norm{Ax}}{\norm{x}}$, 故不妨限制$\norm{x} = 1$, 从而:
	\begin{equation}
		\norm{Ax}_p
		\xrightarrow{\ p\to\infty\ }
			\max\abs{A_{ij} x^j}
		\le \max_i \sum_j \abs{A_{ij}}
	\end{equation}
	当$x^j = \operatorname{sign} A_{ij}$时取到等号。也就是说,$\norm{A}_\infty$即为矩阵的行和最大值。
\subsection{矩阵的欧式模}
	欧式模由于和线性空间上的标准内积一致,因此有额外的优良性质。例如,对$\mbb{C}$上的幺正矩阵$U$而言,$\forall\ x$,\,\ 均有:
	\begin{equation}
	\begin{aligned}
		p = 2,\quad
		\norm{Ux}^2
		= x^\dagger U^\dagger U x
		&= x^\dagger U U^\dagger x
			= \norm{U^\dagger x}^2, \\
		&= x^\dagger x
			= \norm{x}^2,
	\end{aligned}
	\end{equation}
	因此$\norm{U}_2 = \norm{U^\dagger}_2 = 1$. 类似有
		$\norm{(UA)\,x}_2
			= \norm{U(Ax)}_2
			= \norm{Ax}_2$, 
	故$\norm{UA}_2 = \norm{U}_2$. 而矩阵的条件数可一般性地表示为$K_p(A) = \norm{A}_p\,\norm{A^{-1}}_p$, 故:
	\begin{equation}
		K_2(A) = K_2(UA)
	\end{equation}
\subsection{$A$的$\infty$模条件数$K_{\infty}(A)$}
	对于前述例子$A$, 其行和最大值在第一行取到,即:
		$\norm{A}_\infty = n$. 
	类似地,$\norm{A^{-1}}_\infty = 1 + 2^{n-1} - 1
		= 2^{n-1}$, 从而有:
	\begin{equation}
		K_\infty(A)
		= \norm{A}_\infty\,\norm{A^{-1}}_\infty
		= n\,2^{n-1}
	\end{equation}
%\morewhite
\end{document}