\documentclass[preview,10pt,border=8pt]{standalone}
%Citation
	\usepackage[sort&compress,numbers]{natbib}
	\newcommand{\supercite}[1]{\textsuperscript{\,%
		[\citenum{#1}]}}
	\let\fancycite\cite
	\renewcommand{\cite}[1]{\textup{\fancycite{#1}}}
	\renewcommand\refname{参考文献}
%Chinese
	\usepackage[UTF8,heading=false,fontset=fandol]{ctex}
	\xeCJKsetup{underdot = {
		boxdepth=0pt, format=\huge, depth=.4em
	}}
	\newcommand{\cjkdot}[1]{\CJKunderdot{#1}}
%ParagraphSetting
	\setlength{\parskip}{.3\baselineskip}
	\usepackage[defaultlines=2,all]{nowidow}
	\postdisplaypenalty=300
%Presettings
	\usepackage[table]{xcolor}
	\usepackage{graphicx}
	\usepackage[font={sf}]{caption}
	\usepackage[above]{placeins}
%MathSetting
	\let\latexointop\ointop
	\usepackage{amsmath,amssymb,nicefrac,extarrows,eqnarray}%amsthm
	\usepackage{bm,mathrsfs,mathtools,physics}
	\usepackage{wasysym,stmaryrd,esint,siunitx}
	\usepackage{enumitem,stackengine,titling,varwidth}
	\usepackage{tikz,tikz-cd,tkz-euclide}
	\usepackage{resizegather,empheq}
	\usetagform{default}
	\usepackage{calligra,romannum,dsfont,fourier-orns}
	\usepackage{ccicons}
	\let\ointop\undefined
	\let\ointop\latexointop%oint unchanged by esint
	\DeclareMathAlphabet{\mathcalligra}{T1}{calligra}{m}{n}
	\DeclareFontShape{T1}{calligra}{m}{n}{<->s*[2.2]callig15}{}
	\DeclarePairedDelimiter\ave{\langle}{\rangle}
	\DeclarePairedDelimiterX\inprod[2]{\langle}{\rangle}{#1,#2}
	\newcommand{\scriptr}{\mathcalligra{r}\,}
	\newcommand{\rvector}{\pmb{\mathcalligra{r}}\,}
	\newcommand\inlineeqno{\stepcounter{equation}\ (\theequation)}
	\newcommand{\mbb}[1]{\mathbb{#1}}
	\newcommand{\mrm}[1]{\mathrm{#1}}
	\newcommand{\mcal}[1]{\mathcal{#1}}
	\newcommand{\tup}[1]{\textup{#1}}
	\newcommand*\bbox[1]{\fbox{\hspace{1em}\addstackgap[5pt]{#1}\hspace{1em}}}
	\newcommand\scalemath[2]{\scalebox{#1}{\mbox{\ensuremath{\displaystyle #2}}}}
	\newcommand\raisemath[2]{\raisebox{#1\depth}{${#2}$}}
	\empheqset{box=\bbox}
	\numberwithin{equation}{section}
%PageSetting
	\usepackage{array,booktabs,tabularx}
	\usepackage[colorlinks=true,linkcolor=blue]{hyperref}
	\usepackage[vmargin=4cm,hmargin=3cm,%
		footnotesep=\baselineskip]{geometry}
	\usepackage[bottom,splitrule]{footmisc}
	\newcolumntype{C}[1]{>{\hsize=#1\hsize%
		\centering\arraybackslash}X}
	\setlist{itemsep=0pt,topsep=0pt,labelindent=\parindent,leftmargin=0pt,itemindent=*}
	\setlength{\footnotesep}{2.5\parskip}
	\newfontfamily\signature{Vladimir Script}
	\newcommand{\newparagraph}{\pagebreak[3]\noindent%
	%\hrulefill%
	\hfil
	~\raisebox{-4pt}[10pt][10pt]{\leafright~\signature \qquad~\leafleft}~ %
	%\hrulefill%
	\par
	}
%Specials
	\newcommand{\hodgedual}{\operatorname{\star}}
	\newcommand{\dual}{\ \xlongleftrightarrow{\ \textrm{dual}\ }\ }
	\newcommand{\idty}{\mathds{1}}

% Keep parindent
	\usepackage{etoolbox}
	\edef\parindentkept{\the\parindent}
	\edef\parskipkept{\the\parskip}
	\patchcmd{\preview}
	  {\ignorespaces} %%% \preview ends with \ignorespaces
	  {\parindent\parindentkept%
	  \parskip\parskipkept%
	  \ignorespaces}
	  {}{}
% Standalone extras
	\newcommand{\morewhite}{\vspace{16\baselineskip}
		\newparagraph}
	\newcommand{\onelevelup}{../}
../templates/LaTeX/basics/macros.tex

\appendix
\addtocounter{section}{1}
\begin{document}
	\newcommand{\zfunction}{\mcal{Z}_{00} (1;q^2)}
	考察:
	\begin{align}
		\mcal{Z}_{00} (1;q^2)
		= - \pi &+ \frac{1}{\sqrt{4\pi}}
			\sum_{\,\vb{n}\,\in\,\mbb{Z}^3}
			\frac{e^{q^2 - \vb{n}^2}}
				{\vb{n}^2 - q^2}
		\label{eq:sub1} \tag{$\alpha$} \\[.5ex]
		&+ \frac{\pi}{2}
			\int_{0}^{1} \dd{t} t^{-3/2}
				\pqty{e^{tq^2} - 1}
		\label{eq:sub2} \tag{$\kappa$}\\[1ex]
		&+ \sqrt{\frac{\pi}{4}}\,
			\int_{0}^{1} \dd{t} t^{-3/2} e^{tq^2}
			\sum_{\vb{n}\ne 0}
			\exp\pqty{-\frac{\pi^2}{t}\,\vb{n}^2}
		\label{eq:sub3} \tag{$\beta$}
	\end{align}
	这里我们利用了$\mcal{Y}_{00} = \frac{1}{\sqrt{4\pi}}$. 
	
	首先看最简单的 \eqref{eq:sub2}, 它不含求和、仅为一个带参数积分,但积分在左端点处具有奇性;限制$q^2 \in (0,3)$, 则$tq^2 \in (0,3)$, 此时被积函数可展开为一快速收敛的级数,并逐项积分:
	\begin{equation}
	\begin{aligned}
		\eqref{eq:sub2}
		&= \frac{\pi}{2}
			\int_{0}^{1} \dd{t} t^{-3/2}\,
			\pqty{tq^2 + \frac{1}{2!} \pqty{tq^2}^2
				+ \cdots} \\[.8ex]
		&= \frac{\pi}{2}
			\int_{0}^{1} \dd{t} t^{-3/2}\,
			\sum_{k=1}^\infty
				\frac{t^k q^{2k}}{k!} \\
		&= \pi \sum_{k=1}^\infty
			\frac{1}{2k - 1} \frac{q^{2k}}{k!}
		\label{eq:int_to_sum}
	\end{aligned}
	\end{equation}
	由上述展开可见,被积函数在左端点处以
		$t^{-3/2 + 1} = \frac{1}{\sqrt{t}}$
	规律发散,这对数值积分是很不利的;相反级数展开式中分母有$k!$, 收敛迅速,因此后面可以用部分求和的方式计算这一积分%
		\footnote{感谢王子毓同学的提醒。}。
	据 Stirling 近似,只要$\frac{eq^2}{k} < \frac{1}{2}$, 对应临界
		$k \sim 16$, 
	则余项的衰减将远快于几何级数$\pqty{\frac{1}{2}}^k$; 故有余项:
	\begin{equation}
		\epsilon_k
		< \frac{\pi}{2k + 1}
			\frac{1}{\sqrt{2\pi(k+1)}}
		\pqty{\frac{eq^2}{k + 1}}^k
		= \tilde{\epsilon}_k,\quad
		k \ge 16
	\end{equation}
	\showfootnote
	
	\newparagraph
	注意到$\vb{n}\in\mbb{Z}^3,\,\vb{n}^2 = 0,1,2,3,\dots$, 故$\zfunction$的第一项求和 \eqref{eq:sub1} 在$q^2 = 0,1,2,3,\dots$处发散;
	而结合 \eqref{eq:sub3}, \eqref{eq:int_to_sum}, $\zfunction$的其他成分在上述点处均有界,故$q^2 = 0,1,2,3,\dots$是$\zfunction$的奇点。
	相应地,由于$q^2 <3$, 故\textit{在充分接近奇点的区域内},\eqref{eq:sub1} 中的求和至多只需计算到$\vb{n}^2 = 3$. 
	
	对于非奇异的区域,分别考察 \eqref{eq:sub1}, \eqref{eq:sub3} 之求和的截断误差,分别用$\epsilon_\alpha, \epsilon_\beta$标记;在上述分析的基础上,设截断阶数$m^2 \ge 3$, 限定$q^2 \in (0,3)$, 首先有余项:
	\begin{equation}
	\begin{aligned}
		\epsilon_\alpha
		&\equiv \sum_{\vb{n}^2 > m^2}
				\frac{e^{q^2 - \vb{n}^2}}
					{\vb{n}^2 - q^2} \\[.5ex]
		&< \int_{m}^\infty
				\dd{r} 4\pi r^2
				\frac{e^{q^2 - r^2}}{r^2 - q^2}
		= \int_{\frac{m}{q}}^\infty
				\dd{r} 4\pi r^2
				\frac{e^{1 - r^2}}{r^2 - 1} \\[.5ex]
		&< 2\pi e
			\int_{\frac{m}{q}}^\infty
				\dd{r} r\,e^{- r^2}
		= \pi\,e\cdot e^{- \frac{m^2}{q^2}}
		\equiv \tilde{\epsilon}_\alpha
			\pqty{\frac{m}{q}}
	\end{aligned}
	\end{equation}
	其中$q = \sqrt{q^2}$; 注意,上面的“$<$”均是严格成立的;最后一步的放大实际上是相当狠的,将分母上的$(r^2 - 1) \to 2r$, 以保证满足精度要求。
	
	类似地,\eqref{eq:sub3} 中,
	\begin{equation}
	\begin{aligned}
		\epsilon_\beta 
		&\equiv \sum_{\vb{n}^2 > m^2}
			\exp\pqty{
				-\frac{\pi^2}{t}\,\vb{n}^2} \\[.5ex]
		&< \int_{m}^\infty
			\dd{r} 4\pi r^2
			\exp\pqty{-\frac{\pi^2}{t}\,r^2}
		= \frac{4t^{3/2}}{\pi^2}
			\int_{\pi m/\sqrt{t}}^\infty
				\dd{r} r^2 e^{-r^2} \\[.5ex]
		&< \frac{4t^{3/2}}{\pi^2}
		\pqty{\frac{\pi m}{\sqrt{t}}}^{\frac{2}{3}}
			\int_{\pi m/\sqrt{t}}^\infty
				\dd{r} r^3 e^{-r^2}
		= 4t^{\frac{7}{6}}
			\pqty\bigg{\frac{m}{\pi^2}}^{\frac{2}{3}}
			\int_{\pi m/\sqrt{t}}^\infty
				\dd{r} r^3 e^{-r^2} \\[.35ex]
		&= 2t^{\frac{1}{6}}
			\pqty{t + \pi ^2 m^2}
			\pqty\bigg{\frac{m}{\pi^2}}^{\frac{2}{3}}
			e^{-\frac{\pi ^2 m^2}{t}} \\
		&< 2\,\pqty{1 + \pi ^2 m^2}
			\pqty\bigg{\frac{m}{\pi^2}}^{\frac{2}{3}}
			e^{-\frac{\pi ^2 m^2}{t}}
		\equiv \tilde{\epsilon}_\beta
			\pqty{m, t}
	\end{aligned}
	\end{equation}
	可见$\tilde{\epsilon}_\beta$随$m$的衰减规律大致为
		$m^{2 + \frac{2}{3}}
			e^{-\frac{\pi ^2 m^2}{t}}$, 
	还算是比较迅速的。此外,\textit{表观上} \eqref{eq:sub3} 中的积分在 $t\to 0$处存在奇点,与 \eqref{eq:sub2} 相似;但实际有
		$e^{-\frac{\pi ^2}{t}\vb{n}^2}$
	的压低,被积函数在$t\to 0$时趋于零,并不存在实际的奇点。因此,与 \eqref{eq:sub2} 不同,\eqref{eq:sub3} 的积分可以用数值积分方法很好地求出。
	
	\newparagraph
	综上可得,部分求和贡献的误差为:
	\begin{equation}
	\begin{aligned}
		\epsilon(m,m',k,q^2)
		&= \frac{1}{\sqrt{4\pi}}\,
			\epsilon_\alpha\pqty{\frac{m}{q}}
		+ \epsilon_k
		+ \sqrt{\frac{\pi}{4}}
			\int_{0}^{1} \dd{t} t^{-3/2} e^{tq^2}
			\epsilon_\beta(m',t) \\[.8ex]
		&< \tilde{\epsilon}_k
		+ \frac{1}{\sqrt{4\pi}}\,
			\pi\,e\cdot e^{- \frac{m^2}{q^2}}
		+ \sqrt{\frac{\pi}{4}}
			\int_{0}^{1} \dd{t} t^{-3/2} e^{tq^2}
			\tilde{\epsilon}_\beta(m',t) \\
		&\sim \tilde{\epsilon}_k
		+ \sqrt{\frac{\pi}{4}}\,\pqty{\,
			e\cdot e^{- \frac{m^2}{q^2}}
		+ 2\,\pi ^2 m'^2
			\pqty\bigg{\frac{m'}{\pi^2}}^{\frac{2}{3}}
			\int_{0}^{1} \dd{t} t^{-3/2} 
			\exp\pqty{tq^2 - \frac{\pi^2 m'^2}{t}}} \\
		&\equiv \tilde{\epsilon}(m,m',k,q^2)
	\end{aligned}
	\end{equation}
	这里$\tilde{\epsilon}$是放大得到的严格上限,$m,m'$分别是两个对$\vb{n}$求和的截断阶数。由前述分析可知,宜有$m \ge\nolinebreak \sqrt{3} \approx 2$, $m'$的情况暂时不清楚。
%\morewhite
\end{document}
