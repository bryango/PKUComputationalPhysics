\documentclass[preview,10pt,border=8pt]{standalone}
%Citation
	\usepackage[sort&compress,numbers]{natbib}
	\newcommand{\supercite}[1]{\textsuperscript{\,%
		[\citenum{#1}]}}
	\let\fancycite\cite
	\renewcommand{\cite}[1]{\textup{\fancycite{#1}}}
	\renewcommand\refname{参考文献}
%Chinese
	\usepackage[UTF8,heading=false,fontset=fandol]{ctex}
	\xeCJKsetup{underdot = {
		boxdepth=0pt, format=\huge, depth=.4em
	}}
	\newcommand{\cjkdot}[1]{\CJKunderdot{#1}}
%ParagraphSetting
	\setlength{\parskip}{.3\baselineskip}
	\usepackage[defaultlines=2,all]{nowidow}
	\postdisplaypenalty=300
%Presettings
	\usepackage[table]{xcolor}
	\usepackage{graphicx}
	\usepackage[font={sf}]{caption}
	\usepackage[above]{placeins}
%MathSetting
	\let\latexointop\ointop
	\usepackage{amsmath,amssymb,nicefrac,extarrows,eqnarray}%amsthm
	\usepackage{bm,mathrsfs,mathtools,physics}
	\usepackage{wasysym,stmaryrd,esint,siunitx}
	\usepackage{enumitem,stackengine,titling,varwidth}
	\usepackage{tikz,tikz-cd,tkz-euclide}
	\usepackage{resizegather,empheq}
	\usetagform{default}
	\usepackage{calligra,romannum,dsfont,fourier-orns}
	\usepackage{ccicons}
	\let\ointop\undefined
	\let\ointop\latexointop%oint unchanged by esint
	\DeclareMathAlphabet{\mathcalligra}{T1}{calligra}{m}{n}
	\DeclareFontShape{T1}{calligra}{m}{n}{<->s*[2.2]callig15}{}
	\DeclarePairedDelimiter\ave{\langle}{\rangle}
	\DeclarePairedDelimiterX\inprod[2]{\langle}{\rangle}{#1,#2}
	\newcommand{\scriptr}{\mathcalligra{r}\,}
	\newcommand{\rvector}{\pmb{\mathcalligra{r}}\,}
	\newcommand\inlineeqno{\stepcounter{equation}\ (\theequation)}
	\newcommand{\mbb}[1]{\mathbb{#1}}
	\newcommand{\mrm}[1]{\mathrm{#1}}
	\newcommand{\mcal}[1]{\mathcal{#1}}
	\newcommand{\tup}[1]{\textup{#1}}
	\newcommand*\bbox[1]{\fbox{\hspace{1em}\addstackgap[5pt]{#1}\hspace{1em}}}
	\newcommand\scalemath[2]{\scalebox{#1}{\mbox{\ensuremath{\displaystyle #2}}}}
	\newcommand\raisemath[2]{\raisebox{#1\depth}{${#2}$}}
	\empheqset{box=\bbox}
	\numberwithin{equation}{section}
%PageSetting
	\usepackage{array,booktabs,tabularx}
	\usepackage[colorlinks=true,linkcolor=blue]{hyperref}
	\usepackage[vmargin=4cm,hmargin=3cm,%
		footnotesep=\baselineskip]{geometry}
	\usepackage[bottom,splitrule]{footmisc}
	\newcolumntype{C}[1]{>{\hsize=#1\hsize%
		\centering\arraybackslash}X}
	\setlist{itemsep=0pt,topsep=0pt,labelindent=\parindent,leftmargin=0pt,itemindent=*}
	\setlength{\footnotesep}{2.5\parskip}
	\newfontfamily\signature{Vladimir Script}
	\newcommand{\newparagraph}{\pagebreak[3]\noindent%
	%\hrulefill%
	\hfil
	~\raisebox{-4pt}[10pt][10pt]{\leafright~\signature \qquad~\leafleft}~ %
	%\hrulefill%
	\par
	}
%Specials
	\newcommand{\hodgedual}{\operatorname{\star}}
	\newcommand{\dual}{\ \xlongleftrightarrow{\ \textrm{dual}\ }\ }
	\newcommand{\idty}{\mathds{1}}

% Keep parindent
	\usepackage{etoolbox}
	\edef\parindentkept{\the\parindent}
	\edef\parskipkept{\the\parskip}
	\patchcmd{\preview}
	  {\ignorespaces} %%% \preview ends with \ignorespaces
	  {\parindent\parindentkept%
	  \parskip\parskipkept%
	  \ignorespaces}
	  {}{}
% Show footnote
	\let\purefootnote\footnote
	\newcommand{\showfootnote}{}
	\renewcommand{\footnote}[1]{%
		\purefootnote{#1}%
		\renewcommand{\showfootnote}{{%
			\noindent\footnotesize%
			\lefthand\ 
			[\number\value{footnote}]\ 
			\textit{#1}}%
	}}
% Standalone extras
	\newcommand{\morewhite}{\vspace{16\baselineskip}
		\newparagraph}
	\newcommand{\onelevelup}{../}
\newcommand{\mbb}[1]{\mathbb{#1}}
\newcommand{\mrm}[1]{\mathrm{#1}}
\newcommand{\mcal}[1]{\mathcal{#1}}
\newcommand{\tup}[1]{\textup{#1}}
\newcommand{\idty}{\mathds{1}}
%Specials
\newcommand{\hodgedual}{\operatorname{\star}}
\newcommand{\dual}{\ \xlongleftrightarrow{\ \textrm{dual}\ }\ }
\newcommand{\pdd}[1]{\operatorname{\partial_{\mathnormal{#1}}}}


\renewcommand{\thesection}{A}
\renewcommand{\thesubsection}{\roman{subsection}.}
\begin{document}
	与ODE的初值问题(\textit{initial value problem}, IVP)不同,边值问题(\textit{boundary value problem}, BVP)并没有统一的算法,需要具体问题具体分析;若采用有限差分方法,则需要针对方程具体设计相应的差分格式。与之相对,\textit{打靶法}试图将边值问题转化为初值问题,其一般性更强;下面考察打靶法可以方便求解的一类边值问题。一般来说,有方程:
	\begin{equation}
		\dv{\vec{y}}{t}
		= \vec{f}(t, \vec{y}),\quad
		\vec{y} = \pqty\big{
			y^{(1)}, y^{(2)}, \dots, y^{(n)}
		},
	\end{equation}
	
	设求解区间为$\bqty{a, b}$, 定解所需的总约束数目为$n$, 若左端点处有$n-m$个独立约束:
	\begin{equation}
	\begin{gathered}
		a_{11}\, y^{(1)}_a + a_{12}\, y^{(2)}_a + \cdots + a_{1n}\, y^{(n)}_a = a_{10}, \\
		a_{21}\, y^{(1)}_a + a_{22}\, y^{(2)}_a + \cdots + a_{2n}\, y^{(n)}_a = a_{20}, \\
		\cdots
	\end{gathered}
	\quad\Longleftrightarrow\quad
		A\vec{y}_a = \vec{a}_0
	\label{eq:left_constraints}
	\end{equation}
	相应地,右端则应有$m$个约束,记为$B\vec{y}_b = \vec{b}_0$. 
	
	$m = 0$时,边值问题即退化为初值问题;$m \ge 1$时,\eqref{eq:left_constraints} 实际上是一个关于$n$维变量$\vec{y}$的线性方程组,但其系数矩阵只有$n-m$秩。因此,通常情况下,其解是一个$m$维子空间$V_0$. 打靶法的精神即在于,将边值问题转化为\textit{寻找满足右边条件的特定初值},即寻找函数:
	\begin{equation}
	\mcal{F}\colon\quad
	\begin{aligned}
		V_0\ &\longrightarrow\ V_0, \\
		\vec{y}_0\ &\longmapsto\ B\vec{y}_b - \vec{b}_0,
	\end{aligned}
	\end{equation}
	的\cjkdot{零点}。给定$\vec{y}_0 = \operatorname{proj}_{V_0} \vec{y}_a \in V_0$, 求解由$\vec{y}_0$确定的初值问题(IVP),即可给出$\mcal{F}(\vec{y}_0)$, 流程如下:
	\begin{empheq}{equation}
		\vec{y}_0\ 
		\xmapsto{\ 
			\tup{solve \eqref{eq:left_constraints}}\ }\  
		\vec{y}_a\ 
		\xmapsto[\tup{IVP}]{\ 
			\tup{time evolution}\ }\ 
		\vec{y}_b\ 
		\longmapsto\ 
		\mcal{F}(\vec{y}_0)
	\end{empheq}
	
	然而,实际计算时,对高维的$V_0$而言,多元多分量函数$\mcal{F}\colon V_0\to V_0$的求根是十分困难的。此外,$\mcal{F}$函数值本身的计算量也不小,每次计算均需求解一次初值问题;因此,对$m = \dim V_0 > 1$的情况而言,打靶法求解边值问题往往是不切实际的;但是,$m = 1$时,$\mcal{F}$退化为单变量、单分量函数,其根可以方便地迭代(如采用割线法)求出。
	
	综上所述,\textit{一端具有单一约束}的边值问题可以方便地用打靶法求解,此时边界条件为:
	\begin{align}
	\left.
	\begin{gathered}
		a_{11}\, y^{(1)}_a + a_{12}\, y^{(2)}_a + \cdots + a_{1n}\, y^{(n)}_a = a_{10}, \\
		a_{21}\, y^{(1)}_a + a_{22}\, y^{(2)}_a + \cdots + a_{2n}\, y^{(n)}_a = a_{20}, \\
		\cdots\\
		a_{n-1,1}\, y^{(1)}_a + a_{n-1,2}\, y^{(2)}_a + \cdots + a_{n-1,n}\, y^{(n)}_a = a_{n-1,0}, 
	\end{gathered}\qquad
	\right\rbrace&\mathrlap{\ \tup{left}}\\[1ex]
	\begin{gathered}
		\phantom{
		a_{n-1,1}\, y^{(1)}_a + a_{n-1,2}\, y^{(2)}_a + \cdots + a_{n-1,n}\, y^{(n)}_a = a_{n-1,0}, 
		}\\[-\baselineskip]
		b_{1}\, y^{(1)}_b + b_{2}\, y^{(2)}_b + \cdots + b_{n}\, y^{(n)}_b = b_{0},
	\end{gathered}\quad
	\tup{\textasciitilde}\ &\mathrlap{\ \tup{right}}
	\end{align}
	此时初值具有单一自由度$\vec{y}_0 = y_0$,可简单取为某一$y^{(k)}$分量。
%\morewhite
\end{document}
