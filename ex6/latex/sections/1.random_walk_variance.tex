\documentclass[preview,10pt,border=8pt]{standalone}
%Citation
	\usepackage[sort&compress,numbers]{natbib}
	\newcommand{\supercite}[1]{\textsuperscript{\,%
		[\citenum{#1}]}}
	\let\fancycite\cite
	\renewcommand{\cite}[1]{\textup{\fancycite{#1}}}
	\renewcommand\refname{参考文献}
%Chinese
	\usepackage[UTF8,heading=false,fontset=fandol]{ctex}
	\xeCJKsetup{underdot = {
		boxdepth=0pt, format=\huge, depth=.4em
	}}
	\newcommand{\cjkdot}[1]{\CJKunderdot{#1}}
%ParagraphSetting
	\setlength{\parskip}{.3\baselineskip}
	\usepackage[defaultlines=2,all]{nowidow}
	\postdisplaypenalty=300
%Presettings
	\usepackage[table]{xcolor}
	\usepackage{graphicx}
	\usepackage[font={sf}]{caption}
	\usepackage[above]{placeins}
%MathSetting
	\let\latexointop\ointop
	\usepackage{amsmath,amssymb,nicefrac,extarrows,eqnarray}%amsthm
	\usepackage{bm,mathrsfs,mathtools,physics}
	\usepackage{wasysym,stmaryrd,esint,siunitx}
	\usepackage{enumitem,stackengine,titling,varwidth}
	\usepackage{tikz,tikz-cd,tkz-euclide}
	\usepackage{resizegather,empheq}
	\usetagform{default}
	\usepackage{calligra,romannum,dsfont,fourier-orns}
	\usepackage{ccicons}
	\let\ointop\undefined
	\let\ointop\latexointop%oint unchanged by esint
	\DeclareMathAlphabet{\mathcalligra}{T1}{calligra}{m}{n}
	\DeclareFontShape{T1}{calligra}{m}{n}{<->s*[2.2]callig15}{}
	\DeclarePairedDelimiter\ave{\langle}{\rangle}
	\DeclarePairedDelimiterX\inprod[2]{\langle}{\rangle}{#1,#2}
	\newcommand{\scriptr}{\mathcalligra{r}\,}
	\newcommand{\rvector}{\pmb{\mathcalligra{r}}\,}
	\newcommand\inlineeqno{\stepcounter{equation}\ (\theequation)}
	\newcommand{\mbb}[1]{\mathbb{#1}}
	\newcommand{\mrm}[1]{\mathrm{#1}}
	\newcommand{\mcal}[1]{\mathcal{#1}}
	\newcommand{\tup}[1]{\textup{#1}}
	\newcommand*\bbox[1]{\fbox{\hspace{1em}\addstackgap[5pt]{#1}\hspace{1em}}}
	\newcommand\scalemath[2]{\scalebox{#1}{\mbox{\ensuremath{\displaystyle #2}}}}
	\newcommand\raisemath[2]{\raisebox{#1\depth}{${#2}$}}
	\empheqset{box=\bbox}
	\numberwithin{equation}{section}
%PageSetting
	\usepackage{array,booktabs,tabularx}
	\usepackage[colorlinks=true,linkcolor=blue]{hyperref}
	\usepackage[vmargin=4cm,hmargin=3cm,%
		footnotesep=\baselineskip]{geometry}
	\usepackage[bottom,splitrule]{footmisc}
	\newcolumntype{C}[1]{>{\hsize=#1\hsize%
		\centering\arraybackslash}X}
	\setlist{itemsep=0pt,topsep=0pt,labelindent=\parindent,leftmargin=0pt,itemindent=*}
	\setlength{\footnotesep}{2.5\parskip}
	\newfontfamily\signature{Vladimir Script}
	\newcommand{\newparagraph}{\pagebreak[3]\noindent%
	%\hrulefill%
	\hfil
	~\raisebox{-4pt}[10pt][10pt]{\leafright~\signature \qquad~\leafleft}~ %
	%\hrulefill%
	\par
	}
%Specials
	\newcommand{\hodgedual}{\operatorname{\star}}
	\newcommand{\dual}{\ \xlongleftrightarrow{\ \textrm{dual}\ }\ }
	\newcommand{\idty}{\mathds{1}}

% Keep parindent
	\usepackage{etoolbox}
	\edef\parindentkept{\the\parindent}
	\edef\parskipkept{\the\parskip}
	\patchcmd{\preview}
	  {\ignorespaces} %%% \preview ends with \ignorespaces
	  {\parindent\parindentkept%
	  \parskip\parskipkept%
	  \ignorespaces}
	  {}{}
% Show footnote
	\let\purefootnote\footnote
	\newcommand{\showfootnote}{}
	\renewcommand{\footnote}[1]{%
		\purefootnote{#1}%
		\renewcommand{\showfootnote}{{%
			\noindent\footnotesize%
			\lefthand\ 
			[\number\value{footnote}]\ 
			\textit{#1}}%
	}}
% Standalone extras
	\newcommand{\morewhite}{\vspace{16\baselineskip}
		\newparagraph}
	\newcommand{\onelevelup}{../}
\newcommand{\mbb}[1]{\mathbb{#1}}
\newcommand{\mrm}[1]{\mathrm{#1}}
\newcommand{\mcal}[1]{\mathcal{#1}}
\newcommand{\tup}[1]{\textup{#1}}
\newcommand{\idty}{\mathds{1}}
%Specials
\newcommand{\hodgedual}{\operatorname{\star}}
\newcommand{\dual}{\ \xlongleftrightarrow{\ \textrm{dual}\ }\ }
\newcommand{\pdd}[1]{\operatorname{\partial_{\mathnormal{#1}}}}


\renewcommand{\thesection}{A}
\renewcommand{\thesubsection}{\roman{subsection}.}
\begin{document}
\let\vec\nvec % legacy \vec
%随机行走的方差增长
	第$i$步($i = 1,2,\dots,n$)随机行走带来的坐标变动可由矢量$\vec{X}_i$描述;一般来说,它是一个多维随机变量。在此基础上,第$n$步时的坐标可表示为:
	\begin{equation}
		\vec{r}_n = \sum^n_{i = 1} \vec{X}_i
	\end{equation}
	注意到$\vec{X}_i, \vec{X}_j$相互独立且分布一致,自然有$\ave{\vec{r}_n} = n \ave{\vec{X}_i}$; 若$\ave{\vec{X}_i} \ne 0$, 则$\ave{\vec{r}_n}$随步数$n$线性增大。
	
	类似地,
	\begin{gather}
	\begin{aligned}
		\pqty\big{\vec{r}_n - \ave{\vec{r}_n}}^2
		&= \pqty\Bigg{
			\sum^n_{i = 1} \vec{X}_i
		}\cdot\pqty\Bigg{
			\sum^n_{j = 1} \vec{X}_j
		} - 2\vec{r}_n\cdot\ave{\vec{r}_n}
			+ \ave{\vec{r}_n}^2\\[.8ex]
		&= \sum_{i, j} \vec{X}_i\cdot\vec{X}_j
			- 2\vec{r}_n\cdot\ave{\vec{r}_n}
			+ \ave{\vec{r}_n}^2 \\
		&= \sum_{i = j} \vec{X}^2_i 
			+ \sum_{i \ne j}
				\vec{X}_i\cdot\vec{X}_j
			- 2\vec{r}_n\cdot\ave{\vec{r}_n}
			+ \ave{\vec{r}_n}^2,
	\end{aligned}\\[1ex]
	\sigma^2 = \ave*{\pqty\big{
		\vec{r}_n - \ave{\vec{r}_n}
	}^2}
	= n \ave{X^2_i} - \ave{\vec{r}_n}^2
		+ \sum_{i \ne j}
			\ave{\vec{X}_i\cdot\vec{X}_j}
	\end{gather}
	若概率分布关于原点对称(如,各项同性),则
	$\sum_{i \ne j}
			\ave{\vec{X}_i\cdot\vec{X}_j} = 0, 
		\ave{\vec{r}_n} = 0$, 
	相应有$\sigma^2 = n \ave{X^2_i} \propto n$. 
	注意,这一结论适用于任意维数的随机行走。
	
	然而,即便步长固定为1, 若概率分布关于原点\cjkdot{不对称},则
	$\sum_{i \ne j}
			\ave{\vec{X}_i\cdot\vec{X}_j} \ne 0, 
		\ave{\vec{r}_n} \ne 0$, 
	特别地,对一维随机行走而言,若向右、向左概率分别为$p, q,\ p + q = 1$, 则:
	\begin{gather}
		\ave{\vec{r}_n} = n(p - q),\quad
		\ave{\vec{X}_i\cdot\vec{X}_j}
		= p^2 + q^2 - 2pq = (p - q)^2,\quad
		\sum_{i \ne j}
			\ave{\vec{X}_i\cdot\vec{X}_j}
		= n(n - 1)\,(p - q)^2,\\
		\boxed{\ \therefore\ \sigma^2
		= n\,\pqty\big{1 - \pqty{p - q}^2}
		= n\,\mrm{Var}(\vec{X}_i)
		\ \vphantom{\Big|}}
	\end{gather}
	简单起见,这里规定行走的步长恒定为1; 可见$\sigma^2$依然正比于$n$, 但$\abs{p - q}$越大,比例系数越小。若$p = 1$或$q = 1$, 则$\sigma^2 = 0$, 此时随机行走退化为单向规则运动。
	此外,若允许有呆着不走的概率,则$\ave{X^2_i} < 1$, 同样$\sigma^2 = n \ave{X^2_i} \propto n$, 且比例系数小于1. 
%\morewhite
\end{document}
