\documentclass[preview,10pt,border=8pt]{standalone}
%Citation
	\usepackage[sort&compress,numbers]{natbib}
	\newcommand{\supercite}[1]{\textsuperscript{\,%
		[\citenum{#1}]}}
	\let\fancycite\cite
	\renewcommand{\cite}[1]{\textup{\fancycite{#1}}}
	\renewcommand\refname{参考文献}
%Chinese
	\usepackage[UTF8,heading=false,fontset=fandol]{ctex}
	\xeCJKsetup{underdot = {
		boxdepth=0pt, format=\huge, depth=.4em
	}}
	\newcommand{\cjkdot}[1]{\CJKunderdot{#1}}
%ParagraphSetting
	\setlength{\parskip}{.3\baselineskip}
	\usepackage[defaultlines=2,all]{nowidow}
	\postdisplaypenalty=300
%Presettings
	\usepackage[table]{xcolor}
	\usepackage{graphicx}
	\usepackage[font={sf}]{caption}
	\usepackage[above]{placeins}
%MathSetting
	\let\latexointop\ointop
	\usepackage{amsmath,amssymb,nicefrac,extarrows,eqnarray}%amsthm
	\usepackage{bm,mathrsfs,mathtools,physics}
	\usepackage{wasysym,stmaryrd,esint,siunitx}
	\usepackage{enumitem,stackengine,titling,varwidth}
	\usepackage{tikz,tikz-cd,tkz-euclide}
	\usepackage{resizegather,empheq}
	\usetagform{default}
	\usepackage{calligra,romannum,dsfont,fourier-orns}
	\usepackage{ccicons}
	\let\ointop\undefined
	\let\ointop\latexointop%oint unchanged by esint
	\DeclareMathAlphabet{\mathcalligra}{T1}{calligra}{m}{n}
	\DeclareFontShape{T1}{calligra}{m}{n}{<->s*[2.2]callig15}{}
	\DeclarePairedDelimiter\ave{\langle}{\rangle}
	\DeclarePairedDelimiterX\inprod[2]{\langle}{\rangle}{#1,#2}
	\newcommand{\scriptr}{\mathcalligra{r}\,}
	\newcommand{\rvector}{\pmb{\mathcalligra{r}}\,}
	\newcommand\inlineeqno{\stepcounter{equation}\ (\theequation)}
	\newcommand{\mbb}[1]{\mathbb{#1}}
	\newcommand{\mrm}[1]{\mathrm{#1}}
	\newcommand{\mcal}[1]{\mathcal{#1}}
	\newcommand{\tup}[1]{\textup{#1}}
	\newcommand*\bbox[1]{\fbox{\hspace{1em}\addstackgap[5pt]{#1}\hspace{1em}}}
	\newcommand\scalemath[2]{\scalebox{#1}{\mbox{\ensuremath{\displaystyle #2}}}}
	\newcommand\raisemath[2]{\raisebox{#1\depth}{${#2}$}}
	\empheqset{box=\bbox}
	\numberwithin{equation}{section}
%PageSetting
	\usepackage{array,booktabs,tabularx}
	\usepackage[colorlinks=true,linkcolor=blue]{hyperref}
	\usepackage[vmargin=4cm,hmargin=3cm,%
		footnotesep=\baselineskip]{geometry}
	\usepackage[bottom,splitrule]{footmisc}
	\newcolumntype{C}[1]{>{\hsize=#1\hsize%
		\centering\arraybackslash}X}
	\setlist{itemsep=0pt,topsep=0pt,labelindent=\parindent,leftmargin=0pt,itemindent=*}
	\setlength{\footnotesep}{2.5\parskip}
	\newfontfamily\signature{Vladimir Script}
	\newcommand{\newparagraph}{\pagebreak[3]\noindent%
	%\hrulefill%
	\hfil
	~\raisebox{-4pt}[10pt][10pt]{\leafright~\signature \qquad~\leafleft}~ %
	%\hrulefill%
	\par
	}
%Specials
	\newcommand{\hodgedual}{\operatorname{\star}}
	\newcommand{\dual}{\ \xlongleftrightarrow{\ \textrm{dual}\ }\ }
	\newcommand{\idty}{\mathds{1}}

% Keep parindent
	\usepackage{etoolbox}
	\edef\parindentkept{\the\parindent}
	\edef\parskipkept{\the\parskip}
	\patchcmd{\preview}
	  {\ignorespaces} %%% \preview ends with \ignorespaces
	  {\parindent\parindentkept%
	  \parskip\parskipkept%
	  \ignorespaces}
	  {}{}
% Show footnote
	\let\purefootnote\footnote
	\newcommand{\showfootnote}{}
	\renewcommand{\footnote}[1]{%
		\purefootnote{#1}%
		\renewcommand{\showfootnote}{{%
			\noindent\footnotesize%
			\lefthand\ 
			[\number\value{footnote}]\ 
			\textit{#1}}%
	}}
% Standalone extras
	\newcommand{\morewhite}{\vspace{16\baselineskip}
		\newparagraph}
	\newcommand{\onelevelup}{../}
\newcommand{\mbb}[1]{\mathbb{#1}}
\newcommand{\mrm}[1]{\mathrm{#1}}
\newcommand{\mcal}[1]{\mathcal{#1}}
\newcommand{\tup}[1]{\textup{#1}}
\newcommand{\idty}{\mathds{1}}
%Specials
\newcommand{\hodgedual}{\operatorname{\star}}
\newcommand{\dual}{\ \xlongleftrightarrow{\ \textrm{dual}\ }\ }
\newcommand{\pdd}[1]{\operatorname{\partial_{\mathnormal{#1}}}}


\renewcommand{\thesection}{1.a}
\renewcommand{\thesubsection}{\roman{subsection}.}
\begin{document}
	这里从理论上给出QR分解的算法复杂度。由于我们只关心领头阶的贡献,故下述讨论中均记$(n - \mrm{const.}) \sim n$; 例如,对行、列指标的求和有时只有$(n-1)$项,但为简单起见,下面统一写为$\sum^n$. 此外,假定加减、乘除对运算量的贡献均为 1; 对于现代CPU而言,这一假定基本上是成立的\footnote{%
		参见 \tup{\url{https://stackoverflow.com/a/39720217}}. }。
	\showfootnote
\subsection{Householder方法}
	\begin{itemize}
	\item 逐列循环$\sum_k^n$: 
	\begin{itemize}
	\item 考察对角元及以下元素,构成子空间矢量$x$; 构造$(n-k+1)\sim (n-k)$维Householder变换,大致需进行如下操作:
	\begin{itemize}
	\item 求模方$\norm{x}^2$, 复杂度$\order\big{2(n-k)}$; 
	\item 据$x$构造变换矢量$v$, $\order{1}$; 
	\item 求$\norm{v}^2$, 利用$\norm{x}^2$, 仅需$\order{1}$. 
	\end{itemize}
	这一步共需计算量$\order\big{2(n-k)}$. 
	\item 对$\sim (n-k)$维子空间作用Householder变换$H = \idty - 2vv^\mrm{T}/\norm{v}^2$, 
	\begin{itemize}
	\item 首先看$v^\mrm{T}$, 作用在$(n-k)$维子空间上,$\order\big{2(n-k)^2}$; 
	\item $H = \idty - 2vv^\mrm{T}/\norm{v}^2$, 表面上看还需$\order\big{4(n-k)^2}$; 但实际可对每一行$i$先计算系数$2v_i/\norm{v}^2$, 这样计算量可削减为$\order\big{2(n-k)^2}$. 
	\end{itemize}
	这一步共$\order\big{4(n-k)^2}$. 
	\end{itemize}
	这样便得到了QR分解后的$R$矩阵,
	\begin{equation}
		\textit{计算量} \sim \sum_k^n 4\,(n-k)^2
		\sim \frac{4}{3}\,n^3
	\end{equation}
	某些参考材料给出$\order{\frac{2}{3}\,n^3}$的计算量,那是仅考虑了乘法计算的结果\footnote{参见  \tup{\url{https://en.wikipedia.org/wiki/QR\_decomposition\#Using\_Householder\_reflections}}. }。
	\showfootnote
	
	事实上,到此为止我们已经完成了QR分解,只是没有显式地获得$Q$矩阵;但大部分情况下我们只需知道等价的Householder矢量$v$即可,无需显式地写出$Q$矩阵。
	\item 若要进一步获得显式的$Q$矩阵,只需将Householder矩阵累乘即得;注意到Householder矩阵作用的子空间维数$(n-1), (n-2), \dots$顺次递减,完全类似,有:
	\begin{equation}
		\textit{计算量} \sim \sum_{k'}^n 4\,(k')^2
		\sim \frac{4}{3}\,n^3
	\end{equation}
	即若要显式地得到$Q$, 总计算量为$\order{\frac{8}{3}\,n^3}$. 
	\end{itemize}
\subsection{Givens方法}
	\begin{itemize}
	\item 逐行循环$\sum_i^n$: 
	\begin{itemize}
	\item 对$j < i$逐列循环$\sum_j^i$: 
	\begin{itemize}
	\item 构造转动矩阵,$\order{1}$; 
	\item 作用变换,仅会影响$i,j$两行;且前面的迭代已经保证$k < j < i$时$R_{ik} = R_{ij} = 0$,故只需$\order\big{6(n-j)}$; 这里的6即二维旋转的计算量。
	\item 若还需显式求出$Q$矩阵,还需$\order{6i}$; 注意由于Givens变换局限在$\sim i$维子空间内,故只需$\sim 6i$而非$6n$步计算。
	\end{itemize}
	共$\order\big{6(n-j)} + \order{6i}$. 
	\end{itemize}
	\end{itemize}
	求和可得:
	\begin{equation}
		\textit{计算量} \sim \sum_i^n \sum_j^i
			\pqty\big{6(n-j) + 6i}
		\sim \sum_i^n \pqty{(6ni - 3i^2)
			+ 6i^2}
		\sim 2n^3 + 2n^3
		= 4n^3
	\end{equation}
	
	\newparagraph
	Householder和Givens的计算量(领头阶)综合比较如下:
	\begin{equation*}
	\begin{array}{l@{\hspace{2em}}c@{\hspace{1.5em}}c}
	\toprule
		\textit{方法} & \textit{求$R$} & \textit{同时给出显式$Q$} \\
	\midrule
		\tup{Householder} & \frac{4}{3}\,n^3 & \frac{8}{3}\,n^3 \\
		\tup{Givens} & 2n^3 & 4n^3 \\
	\bottomrule
	\end{array}
	\end{equation*}
	可见,两种方法的复杂度同阶,Givens要更高一些;且对任一种方法而言,若要显式地给出$Q$, 计算量差不多要翻倍。由此可得,最好的办法是用Householder矢量或Givens转动参数储存$Q$, 需要$Q$作用时再根据参数逐次作用。
%\morewhite
\end{document}
